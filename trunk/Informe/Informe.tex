\documentclass[a4paper,10pt]{article}
%\usepackage[latin1]{inputenc} % Paquetes de idioma (otro encoding)
\usepackage[utf8]{inputenc} % Paquetes de idioma
\usepackage[spanish]{babel} % Paquetes de idioma
\usepackage{graphicx} % Paquete para ingresar gráficos
\usepackage{grffile}
\usepackage{hyperref}
\usepackage{fancybox}
\usepackage{amsmath}
\usepackage{amsfonts}
\usepackage{listings}
% Paquetes de macros de Circuitos
%\usepackage{pstricks}
\usepackage{tikz}

% Encabezado y Pié de página
\input{EncabezadoyPie.tex}
% Carátula del Trabajo
\title{ \author{} % Lo pongo para que el warning no moleste :p
\setlength{\unitlength}{1cm} %  Especifica la unidad de trabajo
\thispagestyle{empty}

\begin{picture}(18,0)
\put(0,0){\includegraphics[width=1.5cm, height=3cm]{Imagenes/Logo1.png}}

\put(10.5,0){\includegraphics[width=3cm, height=3cm]{Imagenes/Logo2.png}}

\end{picture}
\\[1.5cm]
\begin{center}
	\textbf{{\Huge Facultad de Ingenier\'ia \\ Universidad de Buenos Aires}}\\[2cm]
	{66.09 Laboratorio de Microcomputadoras}\\[0.5cm]
	{Anteproyecto}\\[2.5cm]
\end{center}

\begin{flushleft}
	\textbf{Integrantes:} \\[1cm]

	\begin{tabular}{|c|c|c|}
		\hline
		\textbf{\normalsize Padr\'on} & \textbf{\normalsize Nombre} & \textbf{\normalsize Email} \\
		\hline
		\normalsize 89579 & \normalsize Torres Feyuk, Nicol\'as R. Ezequiel & \normalsize ezequiel.torresfeyuk@gmail.com \\
		\hline
		\normalsize 90406 & \normalsize Levi Hadid, Lucas Alberto & \normalsize lucaslh9@hotmail.com \\
%		\hline
%		\normalsize ????? & \normalsize Madariaga, Eduardo & \normalsize madariagaedu@gmail.com \\
		\hline
	\end{tabular}
\end{flushleft}
\date{} % Hace que no se imprima la fecha en la cual se compilo el .tex
 }

\begin{document}
	\maketitle % Hace que el título anterior sea el principal del documento
	\newpage

	\tableofcontents % Esta línea genera un indice a partir de las secciones y subsecciones creadas en el documento
	\newpage

	\section{Objetivos}
		El objetivo del presente trabajo práctico consiste en diseñar e implementar un sistema que permita controlar los motores de un autito RC mediante 
		comunicación inalámbrica. 

	\section{Prestaciones Técnicas}
		\begin{itemize}
			\item El autito debe contar con 4 velocidades de funcionamiento, algo similar a una caja de cambios.
			\item La comunicación inalámbrica debe abarcar un rango máximo de 10 metros. 
			\item Se debe tener un control absoluto del auto a través del control remoto.
		\end{itemize} 

	\section{Descripción del Proyecto}
		El proyecto consiste en realizar el sistema de control de un auto de juguete el cual utiliza motores de continua para desplazarse, de forma tal de manejar 
		al mismo de forma remota a través de un control. Dado que lo que se desea es diseñar y construir el sistema y no la parte mecánica, se utiliza en el presente
		proyecto  el chasis y los motores de un autito de juguete. \\
		\indent El vehículo a utilizar consta de dos motores de continua en las ruedas traseras del mismo. De esta forma, para lograr que el vehículo gire tanto
		a la derecha como a la izquierda se debe alimentar a uno de los motores mientras el otro se mantiene fijo. Para lograr que el auto se desplace para adelante
		o atrás solo basta con alimentar a los motores con la misma señal de forma que se muevan para el mismo lado y con la misma potencia. \\
		\indent El auto será controlado a través de un control remoto que se encontrará enviando constantemente una señal al autito. Este último procesará esta señal
		y según lo recibido se moverá en la dirección indicada y cambiará su velocidad a la indicada.
	
		
	\section{Diagrama en Bloque}
		\subsection{Autito}
		\subsection{Control}

	\section{Circuitos Esquemáticos}
		\begin{figure}[!htb]
			\centering
			\includegraphics[width=11cm]{../Esquematicos/EsquematicoControl2.pdf}
			\caption{Diagrama en Bloques del Control Remoto} \label{img003}
		\end{figure}
	\section{Listado de Componentes}
	\section{Software}
	\section{Resultados}
	\section{Conclusiones}

\end{document}
